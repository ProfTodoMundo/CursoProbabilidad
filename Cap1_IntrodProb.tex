% !TEX root = NotasCursoProbabilidad.tex

%===========================================
\section{Introducci\'on}
%===========================================



%===========================================
\subsection{Preliminares}
%===========================================

\begin{Def} \textbf{Conjunto}
Es una colecci\'on de elementos con una o varias propiedades en com\'un, lo denotaremos por $\Omega$.
\end{Def}

\begin{Def} \textbf{Conjunto vac\'io}
Es el conjunto que no tiene elementos, se denota por $\emptyset$.
\end{Def}


\begin{Def} \textbf{Subconjunto}
Sean $A$ y $B$ dos conjuntos no vac\'ios, se dice que $A$ es un subconjunto de $B$ si para cualquier elemento de $A$, este pertenece a $B$, es decir, si $x\in A$ entonces $x\in B$, se denota por $A\subset B$.
\end{Def}

\begin{Def} \textbf{Conjuntos iguales}
Sean $A$ y $B$ dos conjuntos no vac\'ios, se dice que $A$ y $B$ son iguales si $A\subset B$ y $B\subset A$, se denota por $A=B$.
\end{Def}

\begin{Def} \textbf{Complemento}
Sea $A$ un conjunto, el complemento de $A$ se define por $A^{c}=\left\{x\in\Omega | x\notin A\right\}$.
\end{Def}


\begin{Def} \textbf{Uni\'on}
Sean $A$ y $B$ dos conjuntos, la uni\'on de $A$ y $B$ se define como $A\cup B=\left\{x\in\Omega | x\in A\vee  x\in B\right\}$
\end{Def}


\begin{Def} \textbf{Intersecci\'on}
Sean $A$ y $B$ dos conjuntos, la uni\'on de $A$ y $B$ se define como $A\cap B=\left\{x\in\Omega | x\in A\wedge  x\in B\right\}$
\end{Def}

\begin{Def} \textbf{Diferencia}
Sean $A$ y $B$ dos conjuntos, la diferencia de $A$ y $B$ se define como: $A-B=\left\{x\in\Omega | x\in A \wedge x\notin B\right\}$.
\end{Def}


\begin{Def} \textbf{Conjuntos excluyentes}
Dados $A$ y $B$ conjuntos, se dice que son mutuamente excluyentes si la intersecci\'on es el conjunto vac\'io.
\end{Def}


\begin{Def} \textbf{Exhaustivos}
Dados $A$ y $B$ conjuntos, se dice que son exhaustivos si $A\cup B=\Omega$.
\end{Def}

\begin{Teo}\textbf{Operaciones entre conjuntos} Sean $A$ y $B$ dos conjuntos, tales que $A,B,C\subset\Omega$, entonces se cumplen las siguientes propiedades
\begin{itemize}
\item \textbf{Conmutatividad}: $A\cup B = B\cup A$ y $A\cap B = B\cap A$.
\item \textbf{Asociatividad de la uni\'on}: $\left(A\cup B\right)\cup C = A\cup\left(B\cup C\right)$
\item \textbf{Asociatividad de la intersecci\'on}: $\left(A\cap B\right)\cap C = A\cap\left(B\cap C\right)$
\item \textbf{Distributividad de la intersecci\'on sobre la uni\'on}: $A\cap\left(B\cup C\right)=\left(A\cap B\right)\cup\left(A\cap C\right)$
\item \textbf{Distributividad de la uni\'on sobre la intersecci\'on}: $A\cup\left(B\cap C\right)=\left(A\cup B\right)\cap\left(A\cup C\right)$
\item \textbf{Complemento de la uni\'on}: $\left(A\cup B\right)^{c}=A^{c}\cap B^{c}$
\item \textbf{Complemento de la intersecci\'on}: $\left(A\cap B\right)^{c}=A^{c}\cup B^{c}$
\item \textbf{Diferencia sim\'etrica}: $A\triangle B=\left(A\cup B\right)-\left(A\cap B\right)$
\end{itemize}
\end{Teo}

\begin{Teo}
Dado $A$ un conjunto, se cumple que $\left(A^{c}\right)^{c}=A$.
\end{Teo}


\begin{Teo}
Sea $A$ conjunto, $A\subset\Omega$, entonces
\begin{multicols}{3}
\begin{itemize}
\item[i) ] $A\cap\Omega=A$

\item[ii) ] $A\cup\Omega=\Omega$
\item[iii) ] $A\cap\emptyset=\emptyset$ 
\item[iv) ] $A\cup\emptyset=A$ 
\item[v) ] $A\cap A^{c}=\emptyset$

\item[vi) ] $A\cup A^{c}=\Omega$
\item[vii) ] $A\cap A=A$ 
\item[viii) ] $A\cup A=A$ 

\end{itemize}
\end{multicols}
\end{Teo}


\begin{Teo}
Sean $A$ y $B$ conjuntos, entonces
\begin{eqnarray*}
A-B=A\cap B^{c}
\end{eqnarray*}
\end{Teo}


\begin{Teo}
Sean $A,B$ y $C$ conjuntos, entonces:
\item $A = (A \cap B) \cup (A \cap B^{c}).$

\item $A^{c} - B^{c} = B - A.$

\item $A \cap B^{c} = A - (A \cap B).$

\item $A \cup B = A \cup (B \cap A^{c}).$

\item $(A - B) - C = A - (B - C).$

\item $A - (B \cap C) = (A - B) \cup (A - C).$
\end{Teo}

De los conceptos anteriores se tiene lo siguiente:
\begin{Def}Sea $\Lambda$un conjunto de \'indices $\{A_\lambda : \lambda \in \Lambda\} = \{A_\lambda\}$ una colecci\'on de subconjuntos de $\Omega$ indexado por $\Lambda$. El conjunto de puntos que consiste en todos los puntos que pertenecen a $A_{\lambda}$ para al menos un $\lambda$ se le llama \emph{union} de conjuntos $\{A_\lambda\}$ y se denota por
\begin{eqnarray*}
\bigcup_{\lambda \in \Lambda} A_\lambda .
\end{eqnarray*}

El conjunto de puntos que pertenecen a $A_\lambda$ para cualquier valor de $\lambda$, se le llama intersecci\'on de los conjuntos $\{A_\lambda\}$ y se denota por 

\begin{eqnarray*}
\bigcap_{\lambda \in \Lambda} A_\lambda .
\end{eqnarray*}

Si $\Lambda$ es vac\'io, entonces se define
\begin{eqnarray*}
\bigcup_{\lambda \in \Lambda} A_\lambda = \emptyset,\textrm{ y }
\bigcap_{\lambda \in \Lambda} A_\lambda = \Omega.
\end{eqnarray*}
\end{Def}

\begin{Teo}
Sea $\Lambda$ un conjunto de \'indices y $\{A_\lambda\}$ una colecci\'on de subconjuntos de $\Omega$ indexado por $\Lambda$. Entonces
\begin{itemize}
\item
\begin{eqnarray*}
\overline{\bigcup_{\lambda \in \Lambda} A_\lambda}= \bigcap_{\lambda \in \Lambda} \overline{A_\lambda}.
\end{eqnarray*}
\item
\begin{eqnarray*}
\overline{\bigcap_{\lambda \in \Lambda} A_\lambda}=\bigcup_{\lambda \in \Lambda} \overline{A_\lambda}.
\end{eqnarray*}
\end{itemize}
\end{Teo}

\begin{Def} Dos conjuntos $A$ y $B$ de $\Omega$ se dice que son  \emph{mutuamente excluyentes} o \emph{disjuntos} si
\begin{eqnarray*}
A \cap B = \emptyset.
\end{eqnarray*}
Los subconjuntos $A_1, A_2, \dots$ son mutuamente excluyentes si
\begin{eqnarray*}
A_i \cap A_j = \emptyset
\quad \text{para cada } i \neq j.
\end{eqnarray*}
\end{Def}

\begin{Teo}
Si $A$ y $B$ son subconjuntos de $\Omega$, entonces
\begin{itemize}
\item
\begin{eqnarray*}
A = (A\cap B) \cup (A\cap\overline{B}),
\end{eqnarray*}
\item
\begin{eqnarray*}
(A\cap B) \cap (A\cap\overline{B}) = \emptyset.
\end{eqnarray*}
\end{itemize}
\end{Teo}


\begin{Teo}
Si $A \subset B$, entonces
\begin{eqnarray*}
A\cap B = A, \text{ y } A \cup B = B.
\end{eqnarray*}
\end{Teo}



%===========================================
\subsection{Fundamentos}
%===========================================


Sea $\Omega$ un conjunto no vac\'io. Una clase de eventos de $\Omega$ es un conjunto cuyos elementos son subconjuntos de $\Omega$. 

\begin{Propty} Dado  $\Omega$ conjunto no vac\'io, se cumplen las siguientes propiedades: 

\begin{itemize}
\item[(i)] $\Omega \in \mathcal{C}$.

\item[(ii)] Si $A \in \mathcal{C}$ entonces $A^{c} \in \mathcal{C}$.

\item[(iii)] Si $A \in \mathcal{C}$ entonces existen $\left\{B_j : j \in J_n\right\} \subset \mathcal{C}$ ajenos por parejas, tales que

\begin{eqnarray*}
A^{c} = \bigcup_{j \in J_n} B_j.
\end{eqnarray*}

\item[(iv)] Si $A_1, A_2 \in \mathcal{C}$ son tales que $A_1 \subset A_2$, entonces $A_{2} - A_{1} \in \mathcal{C}$.

\item[(v)] Si $\left\{A_{j} : j \in J_n\right\} \subset \mathcal{C}$ entonces
\begin{eqnarray*}
\bigcup_{j \in J_n} A_j \in \mathcal{C}.
\end{eqnarray*}

\item[(vi)] Si $\left\{A_{j} : j \in J_{n}\right\} \subset \mathcal{C}$ entonces
\begin{eqnarray*}
\bigcap_{j \in J_{n}} A_j \in \mathcal{C}.
\end{eqnarray*}

\item[(vii)] Si $\{A_n : n \in \mathbb{N}\} \subset \mathcal{C}$ entonces
\begin{eqnarray*}
\bigcup_{n \in \mathbb{N}} A_n \in \mathcal{C}.
\end{eqnarray*}

\item[(viii)] Si $\left\{A_n : n \in \mathbb{N}\right\} \subset \mathcal{C}$ son tales que
$A_n \subset A_{n+1}$ para todo $n \in \mathbb{N}$, entonces
\begin{eqnarray*}
\bigcup_{n \in \mathbb{N}} A_n \in \mathcal{C}.
\end{eqnarray*}

\item[(ix)] Si $\left\{A_n : n \in \mathbb{N}\right\} \subset \mathcal{C}$ son tales que
$A_{n+1} \subset A_n$ para todo $n \in \mathbb{N}$, entonces
\begin{eqnarray*}
\bigcap_{n \in \mathbb{N}} A_n \in \mathcal{C}.
\end{eqnarray*}
\end{itemize}
\end{Propty}



\begin{Def}\textbf{\'Algebra}

\end{Def}



\begin{Def}\textbf{$\sigma$-\'Algebra}

\end{Def}



\begin{Propty}\textbf{Importantes}


\end{Propty}
