% !TEX root = NotasCursoProbabilidad.tex

%==================================================
\section{Presentaci\'on}
%==================================================

La revisi\'on y actualizaci\'on del programa de estudios de Probabilidad I nace de la necesidad de replantear el enfoque que debe tomar el \'area de Probabilidad y Estad\'istica atendiendo al avance de la tecnolog\'ia y de las herramientas que la ciencia pone a la disposici\'on de la sociedad. La ciencia de datos, el aprendizaje autom\'atico y el uso de software para la automatizaci\'on de tareas son sin lugar a dudas los temas actuales que las y los estudiantes deben comenzar a utilizar como parte fundamental de su formaci\'on acad\'emica.\medskip

El curso de Probabilidad I es el primer contacto que las y los estudiantes tienen con el estudio formal de la Probabilidad, y posteriormente, con la Estad\'istica. Es por esta raz\'on  la importancia de desarrollar de forma expl\'icita los temas y subtemas que son necesarios cubrir a lo largo de diecis\'eis semanas. En esta revisi\'on, adem\'as de los temas de probabilidad que las y los estudiantes deben de cubrir a lo largo del semestre, se discuti\'o la importancia en la utilizaci\'on de alg\'un software especializado de distribuci\'on libre, como por ejemplo, R o Python como un primer acercamiento computacional para el estudio de la probabilidad.\medskip

Como parte fundamental de este trabajo se encuentran la reorganizaci\'on y desarrollo de contenidos con respecto a la versi\'on original. Por ejemplo, el tema de funciones generadoras de momentos se consider\'o importante incorporar al final de la unidad dos. Tambi\'en se agregaron dos unidades destinadas al estudio de las variables aleatorias especiales tanto para el caso discreto y el caso continuo.\medskip

Es importante resaltar que todos los temas originales est\'an considerados en esta nueva versi\'on, adem\'as de que durante el estudio de los temas de probabilidad se est\'a considerando el uso de alg\'un software estad\'istico como una herramienta fundamental en el estudio de esta \'area y como elemento fundamental en la formaci\'on de las y los estudiantes de modelaci\'on matem\'atica.\medskip

El nivel de profundidad con el que se describen los temas y subtemas est\'an descritos en los objetivos espec\'ificos de cada unidad tem\'atica mismos que deben ser considerados como una gu\'ia para quienes impartir\'an el curso. Finalmente se hace notar que la bibliograf\'ia se ha enriquecido a trav\'es de libros que se consideran importantes para su consulta, revisi\'on y estudio.


%==================================================
\subsection{Metodolog\'ia}

Se recomienda que las definiciones, resultados y teoremas sean abordados con rigor matem\'atico y con un enfoque que permita que las y los estudiantes comprendan las demostraciones de ciertos teoremas fundamentales as\'i como de sus aplicaciones a trav\'es de ejemplos espec\'ificos.

Se recomienda trabajar una variedad de ejemplos en distintas \'areas de conocimiento: salud, transporte, telecomunicaciones, servicios, etc., que permitan al estudiante comprender con profundidad los conceptos de probabilidad y de la importancia del uso de alg\'un software estad\'istico como facilitador del conocimiento. Por esta raz\'on se recomienda que el curso sea impartido tanto en aula como en el laboratorio de c\'omputo por lo menos una vez a la semana.

%==================================================
\subsection{Programa del curso}

\begin{enumerate}

\item \textbf{Espacio de probabilidad (5 semanas)}
\begin{enumerate}
  \item[1.1 ] Espacio muestral y eventos
  \item[1.2 ] Elementos de an\'alisis combinatorio
  \begin{enumerate}
    \item[1.2.1 ] Principio de multiplicaci\'on
    \item[1.2.2 ] Ordenaciones
    \item[1.2.3 ] Permutaciones
    \item[1.2.4 ] Combinaciones
  \end{enumerate}
  \item[1.3 ] Definici\'on de probabilidad
  \begin{enumerate}
    \item[1.3.1 ] Axiomas de la funci\'on de probabilidad
    \item[1.3.2 ] Propiedades de la funci\'on de probabilidad
  \end{enumerate}
  \item[1.4 ] Conceptos y teoremas b\'asicos de probabilidad
  \begin{enumerate}
    \item[1.4.1 ] Probabilidad condicional e independencia
    \item[1.4.2 ] Teorema de probabilidad total
    \item[1.4.3] Teorema de Bayes
  \end{enumerate}
\end{enumerate}

\item \textbf{Variables aleatorias (5 semanas)}
\begin{enumerate}
  \item[2.1 ] Definici\'on y ejemplos: caso discreto y continuo
  \item[2.2 ] Funci\'on de probabilidad y funci\'on de densidad
  \item[2.3 ] Funci\'on de distribuci\'on
  \item[2.4 ] Esperanza y varianza
  \begin{enumerate}
    \item[2.4.1 ] Propiedades de la esperanza
    \item[2.4.2 ] Propiedades de la varianza
  \end{enumerate}
  \item[2.5 ] Teorema de cambio de variable
  \item[2.6 ] Funci\'on generadora de momentos
  \item[2.7 ] Funci\'on generadora de probabilidad
\end{enumerate}

\item \textbf{Variables aleatorias discretas especiales (3 semanas)}
\begin{enumerate}
  \item[3.1 ] Distribuci\'on Uniforme
  \item[3.2 ] Distribuci\'on Bernoulli
  \item[3.3 ] Distribuci\'on Binomial
  \item[3.4 ] Distribuci\'on Binomial negativa
  \item[3.5 ] Distribuci\'on Geom\'etrica
  \item[3.6 ] Distribuci\'on Hipergeom\'etrica
  \item[3.7 ] Distribuci\'on Poisson
\end{enumerate}

\item \textbf{Variables aleatorias continuas especiales (3 semanas)}
\begin{enumerate}
  \item[4.1 ] Distribuci\'on Uniforme
  \item[4.2 ] Distribuci\'on Normal
  \begin{enumerate}
    \item[4.2.1 ] Aproximaci\'on de distribuciones a la distribuci\'on normal
  \end{enumerate}
  \item[4.3 ] Distribuci\'on Exponencial
  \item[4.4 ] Distribuci\'on Gamma
  \item[4.5 ] Distribuci\'on Beta
  \item[4.6 ] Otras distribuciones
\end{enumerate}

\end{enumerate}

%==================================================
\subsection{Temas y subtemas / Objetivos espec\'ificos}

\begin{itemize}

\item \textbf{1. Espacio de Probabilidad (Tiempo recomendado: 5 semanas)}: -- El/La estudiante conocer\'a la definici\'on formal de espacio de probabilidad. -- Comprender\'a los conceptos y teoremas b\'asicos de la teor\'ia de probabilidad y los aplicar\'a en c\'alculos de probabilidades espec\'ificas.

\begin{enumerate}
\item[1.1 ] \textbf{ Espacio muestral y eventos}: -- Identificar\'a y podr\'a definir el espacio muestral de un experimento aleatorio y podr\'a definir espacios muestrales tanto discretos como continuos utilizando la Teor\'ia de conjuntos.


\item[1.2 ] \textbf{Elementos de an\'alisis combinatorio}: -- El/La estudiante aplicar\'a las t\'ecnicas de conteo para determinar los resultados y el n\'umero de \'estos en un experimento aleatorio discreto. -- Distinguir\'a en qu\'e casos es conveniente utilizar cada m\'etodo.

\begin{enumerate}
\item[1.2.1 ] Principio de multiplicaci\'on
\item[1.2.2 ] Ordenaciones
\item[1.2.3 ] Permutaciones
\item[1.2.4 ] Combinaciones
\end{enumerate}

\item[1.3 ] \textbf{Definici\'on de probabilidad}: -- Conocer\'a los axiomas que definen una probabilidad. -- Demostrar\'a algunas propiedades de la probabilidad derivadas de los axiomas b\'asicos. -- Identificar\'a y aplicar\'a los axiomas y propiedades de la probabilidad para el c\'alculo de probabilidades espec\'ificas. -- El/La estudiante utilizar\'a R u otro software estad\'istico para realizar simulaciones del experimento adecuado para obtener la aproximaci\'on de la probabilidad de que ocurra un evento espec\'ifico.


\begin{enumerate}
\item[1.3.1 ] Axiomas de la funci\'on de probabilidad
\item[1.3.2 ] Propiedades de la funci\'on de probabilidad
\end{enumerate}

\item[1.4 ] \textbf{Conceptos y teoremas b\'asicos de probabilidad}: -- El/La estudiante conocer\'a la definici\'on de probabilidad condicional y de eventos independientes. -- Conocer\'a y aplicar\'a el teorema de la probabilidad total y el Teorema de Bayes en problemas espec\'ificos.

\begin{enumerate}
\item[1.4.1 ] Probabilidad condicional e independencia
\item[1.4.2 ] Teorema de probabilidad total
\item[1.4.3 ] Teorema de Bayes
\end{enumerate}
\end{enumerate}

\item \textbf{2. Variables aleatorias (Tiempo recomendado: 5 semanas)}: -- Comprender\'a el concepto de variable aleatoria y su importancia en el estudio formal de la probabilidad y de la estad\'istica.

\begin{enumerate}
\item[2.1 ] \textbf{Definici\'on y ejemplos, caso discreto y continuo}: -- Diferenciar\'a entre variables aleatorias discretas y continuas considerando ejemplos ilustrativos de cada caso.


\item[2.2 ] \textbf{Funci\'on de probabilidad y funci\'on de densidad}: -- Conocer\'a la definici\'on y las propiedades de la funci\'on de densidad o de probabilidad de una variable aleatoria.

\item[2.3 ] \textbf{Funci\'on de distribuci\'on}: -- Conocer\'a la definici\'on y las propiedades de la funci\'on de distribuci\'on de una variable aleatoria.


\item[2.4 ] \textbf{Esperanza y Varianza}: -- Conocer\'a la definici\'on, las propiedades de la esperanza y la varianza de una variable aleatoria y podr\'a calcularlas para una variedad de ejemplos discretos o continuos.

\begin{enumerate}
\item[2.4.1 ] Propiedades de la esperanza
\item[2.4.2 ] Propiedades de la varianza
\end{enumerate}

\item[2.5 ] \textbf{Teorema de cambio de variable}: -- Aplicar\'a la t\'ecnica de cambio de variable en ejemplos ilustrativos y determinar\'a la funci\'on de distribuci\'on resultante.

\item[2.6 ] \textbf{Funci\'on generadora de momentos}: -- Conocer\'a las propiedades de la Funci\'on Generadora de Momentos. -- Utilizar\'a la funci\'on generadora de momentos para determinar los momentos de una variable aleatoria.

\item[2.7 ]  \textbf{Funci\'on generadora de probabilidad}: -- Aplicar\'a el concepto de funci\'on generadora de probabilidad para describir distribuciones discretas y sus propiedades.
\end{enumerate}

\item \textbf{3. Variables aleatorias discretas especiales (Tiempo recomendado: 3 semanas)}:  -- Para las distribuciones discretas especiales, el/la estudiante conocer\'a la definici\'on y los par\'ametros que definen a cada una de ellas. -- Ser\'a capaz de calcular probabilidades, esperanzas, varianzas y funciones generadoras de cada distribuci\'on en una gran cantidad de ejemplos. -- Identificar\'a la distribuci\'on discreta que es adecuada para modelar y resolver un problema espec\'ifico. -- Conocer\'a los comandos b\'asicos de R u otro software estad\'istico para visualizar las funciones de probabilidad y de distribuci\'on, as\'i como para realizar c\'alculos y/o simulaciones en ejemplos concretos.


\begin{enumerate}
\item[3.1 ] Distribuci\'on Uniforme
\item[3.2 ] Distribuci\'on Bernoulli
\item[3.3 ] Distribuci\'on Binomial
\item[3.4 ] Distribuci\'on Binomial negativa
\item[3.5 ] Distribuci\'on Geom\'etrica
\item[3.6 ] Distribuci\'on Hipergeom\'etrica
\item[3.7 ] Distribuci\'on Poisson
\end{enumerate}

\item \textbf{4. Variables aleatorias continuas especiales (Tiempo recomendado: 3 semanas)}:  -- Para las distribuciones continuas especiales, el/la estudiante conocer\'a la definici\'on y los par\'ametros que definen a cada una de ellas. -- Ser\'a capaz de calcular probabilidades, esperanzas, varianzas y funciones generadoras de cada distribuci\'on en una gran cantidad de ejemplos.  -- Identificar\'a la distribuci\'on continua que es adecuada para modelar y resolver un problema espec\'ifico. -- Conocer\'a las aproximaciones que existen entre algunas de ellas. -- Conocer\'a los comandos b\'asicos de R u otro software estad\'istico para visualizar las funciones de densidad y de distribuci\'on, as\'i como para realizar c\'alculos y/o simulaciones en ejemplos concretos.


\begin{enumerate}
\item Distribuci\'on Uniforme
\item Distribuci\'on Normal
\begin{enumerate}
\item Aproximaci\'on de distribuciones a la distribuci\'on normal
\end{enumerate}
\item Distribuci\'on Exponencial
\item Distribuci\'on Gamma
\item Distribuci\'on Beta
\item Otras Distribuciones
\end{enumerate}

\end{itemize}


%==================================================
\subsection{Indicadores para la certificaci\'on}\medskip

\begin{itemize}
\item \textbf{Unidad 1:}
\begin{itemize}
  \item Aplicar t\'ecnicas combinatorias y teoremas fundamentales.
\end{itemize}

\item \textbf{Unidad 2:}
\begin{itemize}
  \item Definir variables aleatorias y calcular esperanza y varianza.
\end{itemize}

\item \textbf{Unidad 3:}
\begin{itemize}
  \item Identificar distribuciones discretas y utilizar software estad\'istico.
\end{itemize}

\item \textbf{Unidad 4:}
\begin{itemize}
  \item Identificar distribuciones continuas y aplicar simulaciones.
\end{itemize}
\end{itemize}

\subsection{Evaluaci\'on del Curso}

 el curso consistir\'a en 3 sesiones a la semana en las que se desarrollar\'an y revisar\'an los contenidos del curso, las sesiones ser\'an te\'oricas y se complementar\'an con pr\'acticas en \texttt{R}, a trav\'es de scripts en formato \texttt{Rmd} los cuales se compilar\'an en un \'unico documento al que se le denominar\'a \texttt{Portafolio}. Adem\'as del estudio de los contenidos del curso, el/la estudiante desarrollar\'a a lo largo del curso $4$ biograf\'ias de una lista de $52$ cient\'ificos prominentes en la historia de la ciencia. La evaluaci\'on de los contenidos del curso ser\'a a trav\'es de tres evaluaciones formativas en las cuales se deber\'a de demostrar el dominio de los temas revisados en clase. La asistencia ser\'a importante para poder ser evaluado con estos tres elementos, el requisito de porcentaje de asistencia es del $85\%$, en caso de no contar con este porcentaje m\'inimo de asistencia para poder certificar la materia deber\'a de presentar el examen de certificaci\'on, mismo que ser\'a elaborado por el comit\'e de certificaci\'on.


\textbf{Evaluaci\'on del curso}: La calificaci\'on final del curso se obtiene de la suma de los siguientes porcentajes: \textit{Calificaci\'on Final} $=$ \textit{Portafolio} (30\%) $+$ \textit{Biograf\'ias} ($10\%$) $ +$ \textit{Evaluaciones} ($60\%$).



\vfill
\begin{center}
\includegraphics[width=6cm]{qr_NotasProbabilidad_GitHub.png}

\subsection{Repositorio del curso}\medskip
\vspace{0.5em}
\textit{Repositorio GitHub del curso}
\end{center}

